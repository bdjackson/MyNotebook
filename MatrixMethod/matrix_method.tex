\documentclass[11pt,a4paper]{article}

% \graphicspath{{figures/}}
\usepackage{subfigure}
\usepackage{mathrsfs}
\usepackage{multirow}
\usepackage{longtable}
\usepackage{slashed}
\usepackage{xspace}
\usepackage{lscape}
\usepackage{rotating}
\usepackage{array}
\usepackage{booktabs}
\usepackage{hyperref}
\usepackage{dcolumn}
\usepackage{epstopdf}
\usepackage{enumerate}
\usepackage[section]{placeins}
\usepackage{multicol}
\usepackage{color}
\hypersetup{
colorlinks=true,
linkcolor=red,
citecolor=red,
urlcolor=blue
}
\renewcommand{\topfraction}{0.99}
\renewcommand{\bottomfraction}{0.99}

%\newcommand{\N1}{$\chi_{1}^{0}$ }
\newcommand{\LSP}{$\widetilde{\chi}_{1}^{0}$ }
\newcommand{\CI}{$\widetilde{\chi}_{1}^{\pm}$ }

% Title and author
\title{Matrix Method}
\author{Brett Jackson}
% \author[1]{Brett Jackson}
% \affil[1]{\it University of Pennsylvania, USA}

\begin{document}
  \maketitle

  We will start with the real efficiency and fake rate definitions:
  \[
    \begin{array}{cc}
      r = \frac{N_{R}^{t}}{N_{R}^{l}} &
      f = \frac{N_{F}^{t}}{N_{F}^{l}}
    \end{array}
  \]
  These give the rate at which a real or lepton that has already passed the
  loose selection is identified as a tight lepton.

  We can construct the following matrix equation.
  \[
    \left(
      \begin{array}{c}
        N_T \\
        N_L \\
      \end{array}
    \right)
    =
    \left(
      \begin{array}{cc}
        r & f \\
        1-r & 1-f \\
      \end{array}
    \right)
    \left(
      \begin{array}{c}
        N_{R}^{l} \\
        N_{F}^{l} \\
      \end{array}
    \right)
  \]
  This gives the number of events passing the tihgt and loose lepton definitions
  given a sample with $N_{R}^{l}$ real leptons and $N_{F}^{l}$ fake leptons.
  
  We want to get the number of real and fake events given some number of tight
  and loose events, so we invert this matrix, giving:
  \[
    \left(
      \begin{array}{c}
        N_{R}^{l} \\
        N_{F}^{l} \\
      \end{array}
    \right)
    =
    \frac{1}{r-f}
    \left(
      \begin{array}{cc}
        1-f & -f \\
        -(1-r) & r \\
      \end{array}
    \right)
    \left(
      \begin{array}{c}
        N_T \\
        N_L \\
      \end{array}
    \right)
  \]

  If we do the matrix multiplication, we get
  \[
    N_{R}^{l} = \frac{1}{r-f}\left( (1-f)N_T - fN_L \right)
  \]
  \[
    N_{F}^{l} = \frac{1}{r-f}\left( -(1-r)N_T + rN_L \right)
  \]

  But, wait!  This gives us the number of real and fake events assuming we are
  looking at a loose signal sample.  We need to change basis to the tight signal
  sample.  To do this, we go back to the definition of the $r$ and $f$ rates.  

  Converting to the basis of tight leptons, we get:
  \[
    N_{R}^{t} = r\frac{1}{r-f}\left( (1-f)N_T - fN_L \right)
  \]
  \[
    N_{F}^{t} = f\frac{1}{r-f}\left( -(1-r)N_T + rN_L \right)
  \]

\end{document}  
